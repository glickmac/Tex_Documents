%%%%%%%%%%%%%%%%%%%%%%%%%%%%%%%%%%%%%%%%%
% Daily Laboratory Book
% LaTeX Template
%
% This template has been downloaded from:
% http://www.latextemplates.com
%
% Original author:
% Frank Kuster (http://www.ctan.org/tex-archive/macros/latex/contrib/labbook/)
%
% Important note:
% This template requires the labbook.cls file to be in the same directory as the
% .tex file. The labbook.cls file provides the necessary structure to create the
% lab book.
%
% The \lipsum[#] commands throughout this template generate dummy text
% to fill the template out. These commands should all be removed when 
% writing lab book content.
%
% HOW TO USE THIS TEMPLATE 
% Each day in the lab consists of three main things:
%
% 1. LABDAY: The first thing to put is the \labday{} command with a date in 
% curly brackets, this will make a new page and put the date in big letters 
% at the top.
%
% 2. EXPERIMENT: Next you need to specify what experiment(s) you are 
% working on with an \experiment{} command with the experiment shorthand 
% in the curly brackets. The experiment shorthand is defined in the 
% 'DEFINITION OF EXPERIMENTS' section below, this means you can 
% say \experiment{pcr} and the actual text written to the PDF will be what 
% you set the 'pcr' experiment to be. If the experiment is a one off, you can 
% just write it in the bracket without creating a shorthand. Note: if you don't 
% want to have an experiment, just leave this out and it won't be printed.
%
% 3. CONTENT: Following the experiment is the content, i.e. what progress 
% you made on the experiment that day.
%
%%%%%%%%%%%%%%%%%%%%%%%%%%%%%%%%%%%%%%%%%

%----------------------------------------------------------------------------------------
%	PACKAGES AND OTHER DOCUMENT CONFIGURATIONS
%----------------------------------------------------------------------------------------

\documentclass[idxtotoc,hyperref,openany,oneside]{labbook} % 'openany' here removes the gap page between days, erase it to restore this gap; 'oneside' can also be added to remove the shift that odd pages have to the right for easier reading

\usepackage[ 
  backref=page,
  pdfpagelabels=true,
  plainpages=false,
  colorlinks=true,
  bookmarks=true,
  pdfview=FitB]{hyperref} % Required for the hyperlinks within the PDF
  
\usepackage{booktabs} % Required for the top and bottom rules in the table
\usepackage{float} % Required for specifying the exact location of a figure or table
\usepackage{graphicx} % Required for including images
\usepackage{lipsum} % Used for inserting dummy 'Lorem ipsum' text into the template

\usepackage{listings}
\usepackage{color}

\definecolor{green}{rgb}{0,0.5,0}
\definecolor{gray}{rgb}{0.5,0.5,0.5}
\definecolor{mauve}{rgb}{0.58,0,0.82}
\definecolor{red}{rgb}{0.8,0,0}

\lstset{frame=tb,
  language=Python,
  aboveskip=3mm,
  belowskip=3mm,
  showstringspaces=false,
  columns=flexible,
  basicstyle={\small\ttfamily},
  numbers=none,
  numberstyle=\tiny\color{gray},
  keywordstyle=\color{mauve},
  commentstyle=\color{red},
  stringstyle=\color{green},
  breaklines=true,
  breakatwhitespace=true,
  tabsize=3
}

\newcommand{\HRule}{\rule{\linewidth}{0.5mm}} % Command to make the lines in the title page
\setlength\parindent{0pt} % Removes all indentation from paragraphs

%----------------------------------------------------------------------------------------
%	DEFINITION OF EXPERIMENTS
%----------------------------------------------------------------------------------------

\newexperiment{VF_HMM}{Identifying Virulence Factors in Phages}
%\newexperiment{shorthand}{Description of the experiment}

%---------------------------------------------------------------------------------------

\begin{document}

%----------------------------------------------------------------------------------------
%	TITLE PAGE
%----------------------------------------------------------------------------------------

\frontmatter % Use Roman numerals for page numbers
\title{
\begin{center}
\HRule \\[0.4cm]
{\Huge \bfseries Daily Reflections and Goals \\[0.5cm] \Large} % Degree
\HRule \\[1.5cm]
\end{center}
}
\author{\Huge Cody Glickman \\ \\ \LARGE cody.glickman@ucdenver.edu \\[2cm]} % Your name and email address
\date{Beginning 20 June 2018} % Beginning date
\maketitle

\tableofcontents

\mainmatter % Use Arabic numerals for page numbers

%----------------------------------------------------------------------------------------
%	LAB BOOK CONTENTS
%----------------------------------------------------------------------------------------

\labday{Week of 18 June 2018}



\begin{lstlisting}
print("Hello World")
\end{lstlisting} 

\textbf{Wednesday June 20th} \\
Today was the first day I attended of the Rocky Mountain Biohackathon. I mistook the schedule thinking today was the first day of the actual hacking, however, I was pleasantly surprised by the lineup of excellent speakers. Speakers included Ben Langmead, of Bowtie fame, on pan-RNA seq processing pipelines on the cloud. John Rinn described bar-coding RNA like the first Juicy Fruit gum package and James Taylor, of Galaxy web-server fame, described novel ways to interpret chromatin structure. 
\\
Martin introduced me to assemblthon, optical mapping, and the troubles of short read assembly.
\\
I enjoyed a lovely bike ride around Boulder and returned to the conference for more interesting talks. I did have the misfortune of leaving my sunglasses somewhere between the afternoon session and when I left the conference. Stressed out, I ran over to mincha/maarav at EDOS. 


Goals:
Finish Review Article

\vspace{0.1cm}
\textbf{Thursday June 21st} \\
The first day of the Hackathon began with some background from our team leader Upendra. The group discussed many pieces of the project eventually accepting the path to create a true positive using the protein MS/MS data available. I learned a helpful hint of storing data on a seperate branch to prevent other users from having to download the full scope of the repository. I finished the review article. 

\vspace{0.1cm}
\textbf{Sunday June 23rd} \\
I begin packing for the move tomorrow. 




%----------------------------------------------------------------------------------------

\labday{Week of 25 June 2018}


\textbf{Monday June 25th} \\


\end{document}