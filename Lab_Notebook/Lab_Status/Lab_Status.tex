%%%%%%%%%%%%%%%%%%%%%%%%%%%%%%%%%%%%%%%%%
% Daily Laboratory Book
% LaTeX Template
%
% This template has been downloaded from:
% http://www.latextemplates.com
%
% Original author:
% Frank Kuster (http://www.ctan.org/tex-archive/macros/latex/contrib/labbook/)
%
% Important note:
% This template requires the labbook.cls file to be in the same directory as the
% .tex file. The labbook.cls file provides the necessary structure to create the
% lab book.
%
% The \lipsum[#] commands throughout this template generate dummy text
% to fill the template out. These commands should all be removed when 
% writing lab book content.
%
% HOW TO USE THIS TEMPLATE 
% Each day in the lab consists of three main things:
%
% 1. LABDAY: The first thing to put is the \labday{} command with a date in 
% curly brackets, this will make a new page and put the date in big letters 
% at the top.
%
% 2. EXPERIMENT: Next you need to specify what experiment(s) you are 
% working on with an \experiment{} command with the experiment shorthand 
% in the curly brackets. The experiment shorthand is defined in the 
% 'DEFINITION OF EXPERIMENTS' section below, this means you can 
% say \experiment{pcr} and the actual text written to the PDF will be what 
% you set the 'pcr' experiment to be. If the experiment is a one off, you can 
% just write it in the bracket without creating a shorthand. Note: if you don't 
% want to have an experiment, just leave this out and it won't be printed.
%
% 3. CONTENT: Following the experiment is the content, i.e. what progress 
% you made on the experiment that day.
%
%%%%%%%%%%%%%%%%%%%%%%%%%%%%%%%%%%%%%%%%%

%----------------------------------------------------------------------------------------
%	PACKAGES AND OTHER DOCUMENT CONFIGURATIONS
%----------------------------------------------------------------------------------------

\documentclass[idxtotoc,hyperref,openany]{labbook} % 'openany' here removes the gap page between days, erase it to restore this gap; 'oneside' can also be added to remove the shift that odd pages have to the right for easier reading

\usepackage[ 
  backref=page,
  pdfpagelabels=true,
  plainpages=false,
  colorlinks=true,
  bookmarks=true,
  pdfview=FitB]{hyperref} % Required for the hyperlinks within the PDF
  
\usepackage{booktabs} % Required for the top and bottom rules in the table
\usepackage{float} % Required for specifying the exact location of a figure or table
\usepackage{graphicx} % Required for including images
\usepackage{lipsum} % Used for inserting dummy 'Lorem ipsum' text into the template

\usepackage{listings}
\usepackage{color}

\definecolor{green}{rgb}{0,0.5,0}
\definecolor{gray}{rgb}{0.5,0.5,0.5}
\definecolor{mauve}{rgb}{0.58,0,0.82}
\definecolor{red}{rgb}{0.8,0,0}

\lstset{frame=tb,
  language=Python,
  aboveskip=3mm,
  belowskip=3mm,
  showstringspaces=false,
  columns=flexible,
  basicstyle={\small\ttfamily},
  numbers=none,
  numberstyle=\tiny\color{gray},
  keywordstyle=\color{mauve},
  commentstyle=\color{red},
  stringstyle=\color{green},
  breaklines=true,
  breakatwhitespace=true,
  tabsize=3
}

\newcommand{\HRule}{\rule{\linewidth}{0.5mm}} % Command to make the lines in the title page
\setlength\parindent{0pt} % Removes all indentation from paragraphs

%----------------------------------------------------------------------------------------
%	DEFINITION OF EXPERIMENTS
%----------------------------------------------------------------------------------------

\newexperiment{VF_HMM}{Identifying Virulence Factors in Phages}
\newexperiment{GRAB}{Genomic Retrieval and BLAST Database Creation}
\newexperiment{Active_Prophage}{Classifying predicted prophages as active or degraded}
\newexperiment{CLSI_Database}{Creation of rpoB, HSP65, and 16S Databases}
\newexperiment{CAMI_Filter}{Viral Contig Identification Study}
%\newexperiment{shorthand}{Description of the experiment}

%---------------------------------------------------------------------------------------

\begin{document}

%----------------------------------------------------------------------------------------
%	TITLE PAGE
%----------------------------------------------------------------------------------------

\frontmatter % Use Roman numerals for page numbers
\title{
\begin{center}
\HRule \\[0.4cm]
{\Huge \bfseries Laboratory Update \\[0.5cm] \Large} % Degree
\HRule \\[1.5cm]
\end{center}
}
\author{\Huge Cody Glickman \\ \\ \LARGE cody.glickman@ucdenver.edu \\[2cm]} % Your name and email address
\date{May 11th 2018} % Beginning date
\maketitle

\tableofcontents

\mainmatter % Use Arabic numerals for page numbers

%----------------------------------------------------------------------------------------
%	LAB BOOK CONTENTS
%----------------------------------------------------------------------------------------

\labday{May 2nd, 2018}

\experiment{GRAB}

Updated databases to include Strain level searches. Waiting on ISME Travel Grant Status (Mid-May)

BioRxiv: Resupply manuscript to MS ID 246553 

Comments: Include data about functionality compared to other tools or show with an example dataset.

To-Do: Application for Mac, Windows, Linux using Tkinter in Python

\textbf{Found Github Repository for \href{https://github.com/kblin/ncbi-genome-download}{Something Similar} }


\experiment{CAMI_Filter}

Environments have been set-up and are ready to roll. The binner (MaxBin2) requires the raw reads to calculate abundances needed for binning. The CAMI data \href{https://storage.googleapis.com/cami-data-eu/CAMI_low/RL_S001__insert_270.fq.gz}{website} contains dead links to retrieve reads. Will need to reach out to authors for data.  


\experiment{VF_HMM}

Have established baseline prevalence of virulence genes in phage genomes (See attached HTML). 
Focused on phages with more than 30 complete genomes in reference database (13 Genus types). 
Completed BLAST and HMM searches of Virulence factors against metagenomic contigs inferred from vHMM study. 

\textbf{Next Steps}

\begin{itemize}
\item Identify taxonomy of contigs with virulence factors
\item Gather study metadata
\item Compare metadata to significance of VFs
\end{itemize}



\experiment{CLSI_Database}

Data gathered from locations below, next steps are to make this into a virtualbox with instructions to allow for technician usage. 

Downloading rpoB data from NCBI Nucleotide (Copy and Paste into Search Window) \\

1780 Sequences as of May 1st, 2018

\begin{verbatim}
(("Mycobacterium"[Organism] OR ("Mycobacterium"[Organism]
OR Mycobacterium[All Fields])) AND rpoB[Title]) AND
(bacteria[filter] AND biomol_genomic[PROP] AND
ddbj_embl_genbank[filter] AND ("500"[SLEN] : "5000"[SLEN]))
\end{verbatim}


Downloading HSP65 data from NCBI Nucleotide (Copy and Paste into Search Window) \\

1710 Sequences as of May 1st, 2018

\begin{verbatim}
(("Mycobacterium"[Organism] OR ("Mycobacterium"[Organism]
OR Mycobacterium[All Fields])) AND hsp65[Title]) AND
(bacteria[filter] AND biomol_genomic[PROP] AND
ddbj_embl_genbank[filter] AND ("400"[SLEN] : "5000"[SLEN]))
\end{verbatim}



Downloading \href{https://www.arb-silva.de/no_cache/download/archive/release_132/Exports/}{Silva-ARB Database}: download SilvaSSUParctaxsilvatrunc.fasta.gz \\

See \href{https://www.arb-silva.de/documentation/release-132/}{Silva Release} information



\experiment{Active_Prophage}

Concluded three types of genes indicate lysogenic life cycle (see below). Protein HMMs for these three types have been obtained through PFAM. Additional HMMs may need to be developed with sequenced viruses. 

\begin{itemize}
\item Integrases \\
Involved in phage insertion into host genome
\item ParA - ParB - ParS \\
Involved in extra chromosomal arrangement and replication
\item Repressors \\
Inhibits replication until stimulus 
\end{itemize}

%----------------------------------------------------------------------------------------

\labday{May 11th, 2018}

\experiment{GRAB}

Updated databases to include Strain level searches. ISME Travel Grant Status: Abstract Accepted (Word on Travel Grant Unclear)


\experiment{VF_HMM}

Methodology Questions: 

Gene count or percentage of genes

\textbf{Virome Sampling Steps}

\begin{itemize}
\item Identify taxonomy of contigs
\item Predict genes and identify virulence factors
\item Compare percentage in niche to baseline
\end{itemize}


\experiment{CLSI_Database}

SequenceServer is Awesome!

Data does comprise of identical reads with overlapping labels:

i.e. Mycobacterium massiliense strains containing the same sequence as M. abscessus

%----------------------------------------------------------------------------------------

\labday{June 1st, 2018}

\experiment{GRAB}

Updated databases to include Strain level searches. 

ISME Travel Grant Status: Abstract Accepted
\vspace{0.2cm}
Progress and Publication Goals: 

Hold (Incorporate with GUI and SequenceServer) 

Winter 2018


\experiment{VF_HMM}

NLM Presentation
\vspace{0.2cm}

Progress:

Baseline model completed, pipeline for clinical datasets completed, searching for clinical dataset with longer viral contigs 

\vspace{0.2cm}
Publication Goals: July 2018 


\experiment{CLSI_Database}

Creating a database of rpoB, hsp65, and reference mycobacteria with a user interface

Progress: Completed
\vspace{0.2cm}
Publication Goals: June 2018

\end{document}